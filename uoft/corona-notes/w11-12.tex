\documentclass{article}

\usepackage{amsfonts}
\usepackage{amsmath}
\usepackage{amsthm}
\usepackage{changepage}  % For adjustwidth environment
\usepackage{colortbl}  % Provides the  \arrayrulecolor command.
\usepackage{environ}
\usepackage{framed}  % For leftbar environment
\usepackage[hidelinks]{hyperref}  % For \autoref
\usepackage{listings} \lstset{language=Python}
\usepackage{mathtools}  % For \shortintertext
\usepackage{parskip}  % Don't indent paragraphs; skip instead.
\usepackage{ragged2e}  % FlushLeft environment
\usepackage{varwidth}
\usepackage{url}
\usepackage{xcolor}
\usepackage{graphicx}			% Include only if you want to import images.

% Code style, based on one from CSC236,.
\definecolor{codegray}{rgb}{0.5,0.5,0.5}
\definecolor{codepurple}{rgb}{0.58,0,0.82}
\definecolor{backcolour}{rgb}{0.95,0.95,0.92}
 \lstdefinestyle{mystyle}{
    commentstyle=\color{codepurple},
    keywordstyle=\color{blue},
    %numberstyle=\tiny\color{codegray},
    stringstyle=\color{codepurple},
    basicstyle=\ttfamily\small,
    breakatwhitespace=false,
    breaklines=true,
    captionpos=b,
    keepspaces=true,
    %numbers=left,
    %numbersep=5pt,
    showspaces=false,
    showstringspaces=false,
    showtabs=false,
    tabsize=2,
}
\lstset{style=mystyle}


% Stuff that should already be on slides.
\NewEnviron{bigbox}{{

    \newdimen\slidewidth
    \slidewidth=\linewidth
    % Without this, I get errors beginning "Overfull \hbox (6.79999pt too
    % wide)"
    \advance\slidewidth by -7pt

    \medskip

    \fbox{\parbox{\slidewidth}{\BODY}}

    \medskip

  }}

% Stuff I should write or draw during lecture.
\NewEnviron{writenote}{\vspace{1ex}\begin{leftbar}{\BODY}\end{leftbar}}

\newenvironment{indentone}{\begin{adjustwidth}{2em}{0em}}{\end{adjustwidth}}


% Binary and unary all-uppercase text operators. \mathop gives too little space
% around the binary ones; e.g. \(P \op{AND} Q\) needs lots of space between
% those capital letters.
\newcommand\op[1]{{\ \mathrm{#1}\ }}
\newcommand\uop[1]{{\mathrm{#1}}}

\newcommand\AND{{\op{AND}}}
\newcommand\IFF{{\op{IFF}}}
\newcommand\IMPLIES{{\op{IMPLIES}}}
\newcommand\NOT{{\uop{NOT}}}
\newcommand\OR{{\op{OR}}}
\newcommand\XOR{{\op{XOR}}}

\newcommand\FALSE{{\mathrm{F}}}
\newcommand\TRUE{{\mathrm{T}}}

\DeclareMathOperator\concat{+\!\!+}
\DeclareMathOperator\Bkt{Bkt}
\DeclareMathOperator\leftT{left}
\DeclareMathOperator\rightT{right}

\newcommand\stringLit[1]{{\texttt{"#1"}}}

\newcommand\ComplexNums{{\mathbb{C}}}
\newcommand\Ints{{\mathbb{Z}}}
\newcommand\Nats{{\mathbb{N}}}
\newcommand\Reals{{\mathbb{R}}}
\newcommand\Rationals{{\mathbb{Q}}}

\newcommand\lbkt{{\texttt{[}}}
\newcommand\rbkt{{\texttt{]}}}

\newcommand\powerset{{\mathcal{P}}}

\newcommand\optional[1]{\textcolor[rgb]{0.5,0.5,0.3}{#1}}

% Warning: this resets the colour to black using \arrayrulecolor.
\newcommand{\greyline}{\arrayrulecolor{black!20}\hline\arrayrulecolor{black}}

% Don't put theorem text in italics, because in this document, italics is for
% descriptions of things I should write or have on slides.
\theoremstyle{definition}
\newtheorem{theorem}{Theorem}
\newtheorem*{remark*}{Remark}
\newtheorem*{theorem*}{Theorem}
\newtheorem*{definition*}{Definition}
\newtheorem*{lemma*}{Lemma}

\begin{document}
\begin{center}
\textbf{\Large CSC240 Notes}
\end{center}

% put index here
\newpage

\section*{Week 11}
\subsection{Languages}
\begin{bigbox}
    \begin{definition*}
        A \textit{language} over $\Sigma$ is a subset of $\Sigma^*$. 
    \end{definition*} 
\end{bigbox}
\begin{writenote}
    i.e. $L\subseteq\Sigma^*$ is a language of $\Sigma$.
\end{writenote}

\begin{bigbox}
    \begin{definition*}
        (Kleene Star *)\begin{equation*}
            L^*=\bigcup_{k\in\Nats}L^k
        \end{equation*}
    \end{definition*}
\end{bigbox}
\begin{writenote}
    \begin{itemize}
        \item $L^+=\bigcup_{k\in\Ints^+}L^k$
        \begin{itemize}
            \item Notice that $L^*=L^+\cup\{\lambda\}$, so $L^+=L^*$ iff $\lambda\in L$.
        \end{itemize}
        \item Let $L_0,L_1\subseteq\Sigma^*$ be languages. Then $L_0\cup L_1,L_0\cap L_1$ are languages.
        \item The \textit{complement} of a language $L\subseteq\Sigma^*$, $\bar{L}=\Sigma^*-L$, is also a language
    \end{itemize}
\end{writenote}
\subsection{Regular Expressions}
\textit{Regular Expressions} are a concise way to describe some language. 
\begin{bigbox}
    \begin{definition*}
        Given a finite alphabet $\Sigma$, $R_\Sigma$ denotes the set of regular expressions over $\Sigma$. $R_\Sigma$ is 
        the smallest set of strings such that
        \begin{itemize}
            \item Base case: $\emptyset\in R_\Sigma,\lambda\in R_\Sigma$, for every $a\in\Sigma,a\in R_\Sigma.$
            \item Constructor cases: For every $r,r'\in R_\Sigma,$
            \begin{equation*}
                (r+r')\in R_\Sigma,(r\cdot r')\in R_\Sigma,r^*\in R_\Sigma.
            \end{equation*}
        \end{itemize}
    \end{definition*}
\end{bigbox}

\begin{bigbox}
    Given a regular expression
\end{bigbox}

\end{document}
