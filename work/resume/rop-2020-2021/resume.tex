\documentclass{article}
\usepackage[cm]{fullpage}
\usepackage{color}
\usepackage{hyperref}

\hypersetup{breaklinks=true,%
pagecolor=white,%
colorlinks=true,%
linkcolor=cyan,%
urlcolor=MyDarkBlue}

\definecolor{MyDarkBlue}{rgb}{0,0.0,0.45}

%%%%%%%%%%%%%%%%%%%%%%%%%%
% Formatting parameters  %
%%%%%%%%%%%%%%%%%%%%%%%%%%

\newlength{\tabin}
\setlength{\tabin}{1em}
\newlength{\secsep}
\setlength{\secsep}{0.1cm}

\setlength{\parindent}{0in}
\setlength{\parskip}{0in}
\setlength{\itemsep}{0in}
\setlength{\topsep}{0in}
\setlength{\tabcolsep}{0in}

\definecolor{contactgray}{gray}{0.3}
\pagestyle{empty}

%%%%%%%%%%%%%%%%%%%%%%%%%%
%  Template Definitions  %
%%%%%%%%%%%%%%%%%%%%%%%%%%

\newcommand{\lineunder}{\vspace*{-8pt} \\ \hspace*{-6pt} \hrulefill \\ \vspace*{-15pt}}
\newcommand{\name}[1]{\begin{center}\textsc{\Huge#1}\\\end{center}}
\newcommand{\program}[1]{\begin{center}\textsc{#1}\end{center}}
\newcommand{\contact}[1]{\begin{center}\color{contactgray}{\small#1}\end{center}}

\newenvironment{tabbedsection}[1]{
  \begin{list}{}{
      \setlength{\itemsep}{0pt}
      \setlength{\labelsep}{0pt}
      \setlength{\labelwidth}{0pt}
      \setlength{\leftmargin}{\tabin}
      \setlength{\rightmargin}{\tabin}
      \setlength{\listparindent}{0pt}
      \setlength{\parsep}{0pt}
      \setlength{\parskip}{0pt}
      \setlength{\partopsep}{0pt}
      \setlength{\topsep}{#1}
    }
  \item[]
}{\end{list}}

\newenvironment{nospacetabbing}{
    \begin{tabbing}
}{\end{tabbing}\vspace{-1.2em}}

\newenvironment{resume_header}{}{\vspace{0pt}}


\newenvironment{resume_section}[1]{
  \filbreak
  \vspace{2\secsep}
  \textsc{\large#1}
  \lineunder
  \begin{tabbedsection}{\secsep}
}{\end{tabbedsection}}

\newenvironment{resume_subsection}[2][]{
  \textbf{#2} \hfill {\footnotesize #1} \hspace{2em}
  \begin{tabbedsection}{0.5\secsep}
}{\end{tabbedsection}}

\newenvironment{subitems}{
  \renewcommand{\labelitemi}{-}
  \begin{itemize}
      \setlength{\labelsep}{1em}
}{\end{itemize}}

\newenvironment{resume_employer}[4]{
  \vspace{\secsep}
  \textbf{#1} \\ 
  \indent {\small #2} \hfill {\footnotesize#3 (#4)}
  \begin{tabbedsection}{0pt}
  \begin{subitems}
}{\end{subitems}\end{tabbedsection}}


%%%%%%%%%%%%%%%%%%%%%%%%%%
%     Start Document     %
%%%%%%%%%%%%%%%%%%%%%%%%%%

\begin{document}

\begin{resume_header}
\name{Ziyue Yang}
\program{Student Number. 1004804759}
\contact{ziyue.yang@mail.utoronto.ca \hspace{2cm} +1(647)835-0266 \hspace{2cm}yangzi33.github.io}
\end{resume_header}

\begin{resume_section}{Technical Skills}
  \begin{nospacetabbing}

  \textbf{Programming}  \= Python, Java, C++, JavaScript, SQLite, Swift, Ruby, XML\\*
  \textbf{ML/Data} \> Jupyter Notebook, AWS, R, PyTorch, OpenCV, Keras\\*
  \textbf{Web} \> Django, Rails, React\\*
  \textbf{Mobile Dev} \> Android Studio, Xcode\\*
  \end{nospacetabbing}

\end{resume_section}


\begin{resume_section}{Projects}

 \begin{resume_subsection}[(June 2020 - September 2020 (Expected))]{Deep Neural Networks for Tumor Segmentation}
    \begin{subitems}
      \item Head and neck tumor segmentation model based on CNN. The project will be trained on 
      the \textit{HECKTOR Challenge} training data sets of \textit{MICCAI 2020}. The datasets are planned to be released on June 10th, 2020. 
    \end{subitems}
  \end{resume_subsection}

 \begin{resume_subsection}[(June 2020)]{Console FPV Navigator}
    \begin{subitems}
    \item A command-line based first-person 3D navigator, written in C++.
    \item Used a ray casting algorithm to map from 2D space to 3D for rendering.
    \item Inspired by the engine of game \textit{Wolfenstein 3D}.
    \end{subitems}
  \end{resume_subsection}

   \begin{resume_subsection}[(May 2020 - June 2020)]{UniMart}
    \begin{subitems}
    \item Implemented user models, item views, and trade models in Django to allow C2C trading.
    \item Styled front-end templates with the Bootstrap framework.
    \item Stored and managed data with PostgreSQL.
    \item Deployed application using Heroku.
    \end{subitems}
  \end{resume_subsection}

  \begin{resume_subsection}[(April 2020)]{Agenda}
    \begin{subitems}
      \item Integrated Android GUI with various features (e.g.
      auto-filling while searching calendar events by name).
      \item Implemented features that allow users to modify
      repeating events, as well as operations on user-selected events.
      \item Queried app data using SQLite.
    \end{subitems}
  \end{resume_subsection}

\begin{resume_section}{Experience}
  
  \begin{resume_employer}{Bigtheta}{Project Leader (Remote)}{Toronto, ON}{May 2020 - Present}
    \item Leads a team of entry-level student programmers to create Django-based project.
    \item Introduces software development tools and principles (e.g. git, UML, SOLID) in a lecture-like style.  
    \item Demonstrates front-end and back-end developing, with some use of database. 
  \end{resume_employer}
  
  \begin{resume_employer}{Cohesion}{iOS Developer, Internship}{Guangzhou, Guangdong, China}{Summer 2018}
    \item Contributed to the Swift development of Cohesion App, an iOS office reservation tool. 
    \item Implemented web features using JavaScript.
    \item Front-end testing with various browsers.
  \end{resume_employer}
\end{resume_section}


\end{resume_section}
  
  \begin{resume_section}{Education}
    \begin{resume_subsection}[Toronto, ON (2018--2022 (Expected))]{University of Toronto}
      \begin{subitems}
        \item Honours Bachelor of Science
        \item Statistics Major, Minor in Computer Science and Mathematics
      \end{subitems}
    \end{resume_subsection}
  \end{resume_section}

  \begin{resume_section}{Selected Curricula}
    \begin{resume_subsection}[]{Computer Science}
      Software Design, Data Structures, Convolutional Neural Networks, UNIX Programming, 
      Theory of Computation.
    \end{resume_subsection}
  \begin{resume_subsection}[]{Mathematics and Statistics}
    Linear Algebra, Real Analysis, Multivariable Calculus, Probability.
    \end{resume_subsection}
  \end{resume_section}
\end{document}
